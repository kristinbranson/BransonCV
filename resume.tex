%!TEX TS-program = xelatex
%!TEX encoding = UTF-8 Unicode
% Kristin Branson 20210629
%
% Based on the following template:
% Awesome CV LaTeX Template for CV/Resume
%
% This template has been downloaded from:
% https://github.com/posquit0/Awesome-CV
%
% Original author:
% Claud D. Park <posquit0.bj@gmail.com>
% http://www.posquit0.com
%
% Modifications by:
% Junhao Dong <junhao.dong96@gmail.com>
%
% Template license:
% CC BY-SA 4.0 (https://creativecommons.org/licenses/by-sa/4.0/)
%


%-------------------------------------------------------------------------------
% CONFIGURATIONS
%-------------------------------------------------------------------------------
% A4 paper size by default, use 'letterpaper' for US letter
\documentclass[11pt, letterpaper]{awesome-cv}

% Configure page margins with geometry
\geometry{left=1.4cm, top=.8cm, right=1.4cm, bottom=1.8cm, footskip=.5cm}

% Specify the location of the included fonts
\fontdir[fonts/]

% Color for highlights
% Awesome Colors: awesome-emerald, awesome-skyblue, awesome-red, awesome-pink, awesome-orange
%                 awesome-nephritis, awesome-concrete, awesome-darknight
\colorlet{awesome}{awesome-skyblue}
% Uncomment if you would like to specify your own color
% \definecolor{awesome}{HTML}{CA63A8}

% Colors for text
% Uncomment if you would like to specify your own color
% \definecolor{darktext}{HTML}{414141}
% \definecolor{text}{HTML}{333333}
% \definecolor{graytext}{HTML}{5D5D5D}
% \definecolor{lighttext}{HTML}{999999}

% Set false if you don't want to highlight section with awesome color
\setbool{acvSectionColorHighlight}{true}

% If you would like to change the social information separator from a pipe (|) to something else
\renewcommand{\acvHeaderSocialSep}{\quad\textbar\quad}

\makeatletter
\patchcmd{\@sectioncolor}{\color}{\mdseries\color}{}{}
\makeatother

%-------------------------------------------------------------------------------
%	PERSONAL INFORMATION
%	Comment any of the lines below if they are not required
%-------------------------------------------------------------------------------
% Available options: circle|rectangle,edge/noedge,left/right
% \photo[rectangle,edge,right]{profile}
\name{Kristin}{Branson}
\position{Senior Group Leader}
% \address{address}

% \mobile{(678) 343-1817}
\email{kristinbranson@gmail.com}
\homepage{https://www.janelia.org/lab/branson-lab}
\github{kristinbranson}
% \linkedin{junhaodong}
% \gitlab{gitlab-id}
% \stackoverflow{SO-id}{SO-name}
\googlescholar{g558OVoAAAAJ}{Kristin Branson}
% \twitter{@KristinMBranson}
\mastodon{social.coop}{kristinmbranson}
% \skype{skype-id}
% \reddit{reddit-id}
% \extrainfo{extra informations}

%-------------------------------------------------------------------------------
\begin{document}

% Print the header with above personal informations
% Give optional argument to change alignment(C: center, L: left, R: right)
\makecvheader[C]

% Print the footer with 3 arguments(<left>, <center>, <right>)
% Leave any of these blank if they are not needed
% \makecvfooter
%   {\today}
%   {Junhao Dong~~~·~~~Résumé}
%   {\thepage}

%-------------------------------------------------------------------------------
%	CV/RESUME CONTENT
%	Each section is imported separately, open each file in turn to modify content
%-------------------------------------------------------------------------------
% \input{resume/education.tex}
% \input{resume/skills.tex}
% \input{resume/experience.tex}
% \input{resume/projects.tex}

\cvsection{Summary}
\vspace{\acvSectionContentTopSkip}\\
I develop new and impactful ways to use computer vision (CV) and machine learning (ML) to gain insight into scientific questions. I do this by 1) finding new scientific questions that can be framed as CV\&ML problems, 2) engineering practical and integrative solutions, and 3) making these systems usable by others so that these methods are broadly adopted and applied to new problems. I was a pioneer of video-based analysis of animal behavior, and my work helped establish this as an important and ubiquitous technique in neuroscience and biology more generally.

\cvsection{Education}
\begin{cventries}
  \cventry
    {Ph.D., M.S., Computer Science} % Degree
    {University of California, San Diego} % Institution
    {La Jolla, CA} % Location
    {2007, 2002} % Date(s)
    {
      \begin{cvitems} % Description(s) bullet points
      \item Dissertation Title: {\it Tracking Multiple Mice through Severe Occlusions}
      \item Advisors: Serge Belongie and Sanjoy Dasgupta
      \end{cvitems}
    }
    \cventry{A.B., {\it Cum Laude}, Computer Science}
    {Harvard University}
    {Cambridge, MA}
    {2000}
    {}
\end{cventries}
\vspace{-4mm}
\cvsection{Experience}


\begin{cventries}
\cventryfour{HHMI Janelia Research Campus}{Ashburn, VA}{Senior Group Leader}{2017 - PRESENT}{Head of Computation \& Theory}{2017 - 2022}{Group Leader}{2010 - 2017}{}
\cventry{Postdoctoral Researcher}{California Institute of Technology}{Pasadena, CA}{2007 - 2010}{
\begin{cvitems}
  \item Advisors: Pietro Perona and Michael Dickinson
\end{cvitems}
}
\end{cventries}

%\cvsection{Selected Publications}
%\vspace{\acvSectionContentTopSkip}
%\vspace{2mm}
%\begin{cvenum}

\item I.~S. Kwak, D.~Kriegman, and K.~Branson, ``Detecting the starting frame of actions in video,'' {\em WACV}, 2020.
  
\item B.~Sauerbrei, J.-Z. Guo, M.~Mischiati, W.~Guo, M.~Kabra, N.~Verma, B.~Mensch, K.~Branson, and A.~Hantman, ``Cortical pattern generation during dexterous movements is input-driven,'' {\em Nature}, vol.~577 pp.~386--391, 2019.

\item D.~J. Im, H.~Ma, G.~W. Taylor, and K.~Branson,
  ``Quantitatively evaluating {GAN}s with divergences proposed for training,''
  in {\em International Conference on Learning Representations},
  2018.
  
\item A.~A. Robie, J.~Hirokawa, A.~W. Edwards, L.~A. Umayam,
  A.~Lee, M.~L. Phillips, G.~M. Card, W.~Korff, G.~M. Rubin, J.~H. Simpson,
  M.~B. Reiser, and K.~Branson, ``Mapping the neural substrates of behavior,''
  {\em Cell}, vol.~170, no.~2, pp.~393--406, 2017.

\item E.~Eyjolfsdottir, K.~Branson, Y.~Yue, and P.~Perona,
  ``Learning recurrent representations for hierarchical behavior modeling,'' in {\em International Conference on Learning Representations}, 2017.

\item S.~R. Egnor and K.~Branson, ``Computational analysis of behavior,'' {\em Annual Review of Neuroscience}, vol.~39, pp.~217--236, 2016.

\item J.-Z.~Guo, A.~R. Graves, W.~W. Guo, J.~Zheng, A.~Lee,
  J.~Rodriguez-Gonzalez, N.~Li, J.~J. Macklin, J.~W. Phillips, B.~D. Mensh,
  K.~Branson, and A.~Hantman, ``Cortex commands the performance of skilled
  movement,'' {\em e{L}ife}, vol.~4, p.~e10774, 2015.

\item N.~Verma and K.~Branson, ``Sample complexity of learning mahalanobis distance metrics,'' in {\em Advances in Neural Information Processing Systems}, pp.~2584--2592, 2015.
  
\item M.~Kabra, A.~Robie, and K.~Branson, ``Understanding classifier errors by
  examining influential neighbors,'' in {\em Computer Vision and Pattern
  Recognition}, June 2015.

\item Y.~Aso, D.~Sitaraman, T.~Ichinose, K.~R. Kaun, K.~Vogt, G.~Belliart-Gu{\'e}rin, P.-Y. Pla{\c{c}}ais, A.~A. Robie, N.~Yamagata, C.~Schnaitmann, W.~J. Rowell, R.~M. Johnston, T.-T~B. Ngo, N.~Chen, W.~Korff, M.~N. Nitabach, U.~Heberlein, T.~Preat, K.~Branson, H.~Tanimoto, and G.~M. Rubin, ``Mushroom body output neurons encode valence and guide memory-based action selection in {\em {{D}}rosophila},'' {\em eLife}, vol.~3, p.~e04580, 2014.

\item F.~Amat, W.~Lemon, D.~P. Mossing, K.~McDole, Y.~Wan, K.~Branson, E.~W. Myers,  and P.~J. Keller, ``Fast, accurate reconstruction of cell lineages from large-scale fluorescence microscopy data,'' {\em Nature Methods}, vol.~11, no.~9, p.~951, 2014.

\item D.~G. Tervo, M.~Proskurin, M.~Manakov, M.~Kabra, A.~Vollmer, K.~Branson, and
  A.~Y. Karpova, ``Behavioral variability through stochastic choice and its
  gating by anterior cingulate cortex,'' {\em Cell}, vol.~159, no.~1,
  pp.~21--32, 2014.
  
\item M.~Kabra, A.~A. Robie, M.~Rivera-Alba, S.~Branson, and K.~Branson, ``{{JAABA}}:  An interactive machine-learning tool for automatic annotation of animal behavior,'' {\em Nature Methods}, vol.~10, pp.~64--67, 2013.
  
\item K.~Branson, A.~A. Robie, J.~Bender, P.~Perona, and M.~Dickinson, ``High-throughput ethomics in large groups of \emph{{D}rosophila},'' {\em Nature Methods}, vol.~6, pp.~451--457, 2009.

\end{cvenum}

%\vspace{\acvSectionContentTopSkip}

\cvsection{Open-Source Software}
\vspace{\acvSectionContentTopSkip}\\
I lead, co-developed, and co-maintain the following open-source pieces of software:\\
\begin{cvitems}
\item APT: The Animal Part Tracker, \url{http://kristinbranson.github.io/APT/}
  \item BABAM: The Browsable Atlas of Behavior-Anatomy Maps, \url{https://kristinbranson.github.io/BABAM/}
\item BIAS: Basic Image Acquisition Software \url{http://stuff.iorodeo.com/notes/bias/}
\item JAABA: Janelia Automatic Animal Behavior Annotator, \url{http://jaaba.sourceforge.net/}
\item Ctrax: The Caltech Multiple Walking Fly Tracker, \url{http://ctrax.sourceforge.net/}
\end{cvitems}
\vspace{\acvSectionContentTopSkip}

\cvsection{Conferences Organized}
\vspace{\acvSectionContentTopSkip}
\vspace{2mm}
\begin{cvitems}
\item Multi-Agent Behavior Workshop, CVPR, June, 2022. 
\item 4D Cellular Physiology Reimagined: Theory as a Principal Component, September, 2021. 
\item Janelia Conference: Women in Computational Biology, November 2019.
\item Bioimage Computing Workshop, CVPR, June 2019.
\item Society for Neuroscience Virtual Conference on Machine Learning, June 2019.
\item Janelia Junior Scientist Workshop on Machine Learning and Computer Vision, April 2019, October 2017, October 2016, October 2015.
\item Janelia Conference: What Can Machine Learning Do for Neuroscience and Vice-Versa?, November 2010.
\end{cvitems}
\vspace{\acvSectionContentTopSkip}

\cvsection{Honors and Awards}
\vspace{\acvSectionContentTopSkip}
\vspace{2mm}
\begin{cvitems}
\item Selected as one of Cell’s ``40 under 40'' scientists in commemoration of their 40th anniversary, 2015.
\item Faculty of 1000 Recommendation for JAABA, 2013.
\item NASA Graduate Student Researcher Program Fellowship, 2003-2006.
\end{cvitems}
\vspace{\acvSectionContentTopSkip}

\cvsection{Scientific Leadership}
\vspace{\acvSectionContentTopSkip}
\vspace{2mm}
\begin{cvitems}
\item Established the Computation and Theory Research Area at Janelia (10 labs) and served as inaugural Head from 2017-2022. Set the overall vision and direction for computational research at Janelia; recruited, made hiring and renewal decisions, and set budgets for Group Leaders; initiated development of Janelia's GPU cluster; initiated Janelia's Theory Fellow program. 
\item As part of Senior Leadership at Janelia, participated in institute-wide planning and decision-making, 2017-2022.
\item Co-created the Janelia Diversity, Equity, and Inclusion Committee, 2018. Chaired committee 2018-2020. Participated 2020-2022. As part of this committee, we have researched and proposed initiatives to Janelia leadership, invited speakers to present at Janelia, and started a DEI reading group.
\item Started, co-organized, and contributed to Computer Vision and Machine Learning weekly reading group and internal seminar series at Janelia, 2014-present.
\item Started and co-organized the Janelia Computation and Theory Seminar series, bi-weekly, 2018-present. 
\item Participated in and led a Women and Non-binary Scientist Mentoring Group at Janelia, 2018-present.
\item Started and co-organized Janelia Computation and Theory social event to build community at Janelia, 2018-2020. 
\end{cvitems}
\vspace{\acvSectionContentTopSkip}

\cvsection{Outreach}
\vspace{\acvSectionContentTopSkip}
\vspace{2mm}

\begin{cvitems}
\item Directed the Cajal Machine Learning for Neuroscience Summer School (Summer, 2023). 
\item Taught classes on quantitative analysis of fly behavior at the Marine Biology Lab (June-July, 2022) and Howard University, an HBCU (September-November, 2022). 
\item Started and organized ``Hour of Code'' bi-monthly outreach activity for Janelia scientists to share their enthusiasm for and the impact of programming with grade-school children at local elementary schools and libraries, 2017-2020. This was the first student outreach activity at Janelia, and instigated Janelia to build a Community Relations group to support future efforts. 
\item Worked with high school teachers at Loudoun Academies of Science to develop a machine learning course for high school students, 2018. Volunteer to provide project mentorship to students in these classes, 2018-present.
\item Participated in ``Raising Excitement for Science, Engineering, and Technology'' (RESET) Education Outreach Program, 2018-2020, program organized by lab member Alice Robie. 
\item Taught/mentored in several summer and winter schools, including Neuromatch, FENS, Cajal, Cold Spring Harbor, Jackson Laboratory, and FLiACT. Example of a Colab notebook developed for this purpose: \url{https://bit.ly/jaxpose}.
\end{cvitems}
\vspace{\acvSectionContentTopSkip}

%\end{document}

\cvsection{Publications}
\vspace{\acvSectionContentTopSkip}
\vspace{2mm}
\begin{cvenum}
%\item I.~Kwak, D.~Kriegman, and K.~Branson, ``Structured rnns for action start
%  detection,'' {\em arXiv preprint arXiv:????.????}, 2019.

\item M.~Isaacson, J.~Eliason, A.~Nern, E.~Rogers, G.~Lott, T.~Tabachnik, W.~Rowell, A.~Edwards, W.~Korff, G.~Rubin, K.~Branson. ``Small-field visual projection neurons detect translational optic flow and support walking control,'' {\em bioRxiv} 2023:2023-06, 2023

\item H.~Chiu, A.~Robie, K.~Branson, T.~Vippa, S.~Epstein, G.~Rubin, D.~Anderson, C.~Schretter. ``Cell type-specific contributions to a persistent aggressive internal state in female Drosophila,'' {\em eLife} 12:RP88598, 2023. 

\item J.~J. Sun, A.~Ulmer, D.~Chakraborty, B.~Geuther, E.~Hayes, H.~Jia, V.~Kumar, Z.~Partridge, A.~Robie, C.~E.~Schretter,  C. Sun, K. Sheppard, P. Uttarwar, P. Perona, Y. Yue, K. Branson, and A. Kennedy. ``The MABe22 Benchmarks for Representation Learning of Multi-Agent Behavior,'' {\em arXiv:2207.10553}, 2022. 

\item J.~Z. Guo, B. Sauerbrei, J. Cohen, M. Mischiatti, A. Graves, K. Branson, and A. Hantman. ``Disrupting cortico-cerebellar communication impairs dexterity'', {\em eLife} 2021;10:e65906, 2021.
  
\item C.~E. Schretter, Y. Aso, M. Dreher, A.~A. Robie, M.-J. Dolan, N. Chen, M. Ito, T. Yang, R. Parekh, K. Branson, and G.~M. Rubin, ``Neuronal circuitry underlying female aggression in {\em {{D}}rosophila}, {\em eLife}, eLife 2020;9:e58942, 2020.
  
\item D.~J. Im, I.~S. Kwak, and K.~Branson, ``Evaluation metrics for behavior modeling,'' arXiv:2007.12298, 2020.
  
\item I.~S. Kwak, D.~Kriegman, and K.~Branson, ``Detecting the starting frame of actions in video,'' {\em WACV}, 2020.

\item B.~Sauerbrei, J.-Z. Guo, M.~Mischiati, W.~Guo, M.~Kabra, N.~Verma, B.~Mensch, K.~Branson, and A.~Hantman, ``Cortical pattern generation during dexterous movements is input-driven,'' {\em Nature}, vol.~577 pp.~386--391, 2019.

\item D.~J. Im, S.~Prakhya, J.~Yan, S.~Turaga, and K.~Branson, ``Importance weighted
  adversarial variational autoencoders for spike inference from calcium imaging
  data,'' {\em CoRR}, vol.~abs/1906.03214, 2019.

\item J.-Z. Guo, B.~Sauerbrei, J.~D. Cohen, M.~Mischiati, A.~Graves, F.~Pisanello,
  K.~Branson, and A.~W. Hantman, ``The pontine nuclei are an integrative
  cortico-cerebellar link critical for dexterity,'' {\em bioRxiv}, 2019.

\item J.~M. Ache, S.~Namiki, A.~Lee, K.~Branson, and G.~M. Card, ``Context-dependent
  decoupling of sensory and motor circuits underlies behavioral flexibility,''
  {\em Nature Neuroscience}, vol.~22, no.~7, 2019.

\item D.~J. Im, N.~Verma, and K.~Branson, ``Stochastic neighbor embedding under
  f-divergences,'' {\em CoRR}, vol.~abs/1811.01247, 2018.

\item Daniel Jiwoong~Im, H.~Ma, G.~W. Taylor, and K.~Branson,
  ``Quantitatively evaluating {GAN}s with divergences proposed for training,''
  in {\em International Conference on Learning Representations},
  2018.

\item K.~McDole, L.~Guignard, F.~Amat, A.~Berger, G.~Malandain, L.~A. Royer, S.~C.
  Turaga, K.~Branson, and P.~J. Keller, ``In toto imaging and reconstruction of
  post-implantation mouse development at the single-cell level,'' {\em Cell},
  vol.~175, no.~3, pp.~859--876, 2018.

\item I.~F. Rodriguez, R.~Megret, R.~Egnor, K.~Branson, J.~L. Agosto, T.~Giray, and
  E.~Acuna, ``Multiple animals tracking in videousing part affinity fields,''
  in {\em Visual Observation and Analysis of Vertebrate And Insect Behavior},
  2018.

\item I.~F. Rodriguez, K.~Branson, E.~Acuna, J.~L. Agosto-Rivera, T.~Giray, and
  R.~Megret, ``Honeybee detection and pose estimation using convolutional
  neural networks,'' in {\em Reconnaissance des Formes, Image, Apprentissage et
  Perception}, 2018.

\item K.~Branson, ``A deep (learning) dive into a cell,'' {\em Nature Methods},
  vol.~15, no.~4, p.~253, 2018.

\item A.~A. Robie, J.~Hirokawa, A.~W. Edwards, L.~A. Umayam,
  A.~Lee, M.~L. Phillips, G.~M. Card, W.~Korff, G.~M. Rubin, J.~H. Simpson,
  M.~B. Reiser, and K.~Branson, ``Mapping the neural substrates of behavior,''
  {\em Cell}, vol.~170, no.~2, pp.~393--406, 2017.

\item R.~Sen, M.~Wu, K.~Branson, A.~Robie, G.~M. Rubin, and B.~J. Dickson,
  ``Moonwalker descending neurons mediate visually evoked retreat in
  {D}rosophila,'' {\em Current Biology}, vol.~27, no.~5, pp.~766--771, 2017.

\item A.~A. Robie, K.~M. Seagraves, S.~R. Egnor, and K.~Branson, ``Machine vision
  methods for analyzing social interactions,'' {\em Journal of Experimental
  Biology}, vol.~220, no.~1, pp.~25--34, 2017.

\item D.~J. Im, M.~Tao, and K.~Branson, ``An empirical analysis of deep network loss
  surfaces,'' {\em CoRR}, vol.~abs/1612.04010, 2016.

\item E.~Eyjolfsdottir, K.~Branson, Y.~Yue, and P.~Perona,
  ``Learning recurrent representations for hierarchical behavior modeling,'' in
  {\em International Conference on Learning Representations},
  2017.

\item S.~R. Egnor and K.~Branson, ``Computational analysis of behavior,'' {\em Annual
  Review of Neuroscience}, vol.~39, pp.~217--236, 2016.

\item J.-Z. Guo, A.~R. Graves, W.~W. Guo, J.~Zheng, A.~Lee,
  J.~Rodriguez-Gonzalez, N.~Li, J.~J. Macklin, J.~W. Phillips, B.~D. Mensh,
  K.~Branson, and A.~Hantman, ``Cortex commands the performance of skilled
  movement,'' {\em e{L}ife}, vol.~4, p.~e10774, 2015.

\item K.~Branson and J.~Freeman, ``Imaging the neural basis of locomotion,'' {\em
  Cell}, vol.~163, no.~3, pp.~541--542, 2015.

\item W.~C. Lemon, S.~R. Pulver, B.~H{\"o}ckendorf, K.~McDole, K.~Branson,
  J.~Freeman, and P.~J. Keller, ``Whole-central nervous system functional
  imaging in larval {\em {{D}}rosophila},'' {\em Nature Communications}, vol.~6, p.~7924,
  2015.

\item T.~Ohyama, C.~M. Schneider-Mizell, R.~D. Fetter, J.~V. Aleman, R.~Franconville, M.~Rivera-Alba, B.~D. Mensh, K.~M. Branson, J.~H. Simpson, J.~W. Truman, {\em et~al.}, ``A multilevel multimodal circuit enhances action selection in {\em {{D}}rosophila},'' {\em Nature}, vol.~520, no.~7549, p.~633, 2015.

\item N.~Verma and K.~Branson, ``Sample complexity of learning mahalanobis distance metrics,'' in {\em Advances in Neural Information Processing Systems}, pp.~2584--2592, 2015.

\item M.~Kabra, A.~Robie, and K.~Branson, ``Understanding classifier errors by
  examining influential neighbors,'' in {\em Computer Vision and Pattern
  Recognition}, June 2015.

\item K.~Branson, ``Distinguishing seemingly indistinguishable animals with computer
  vision,'' {\em Nature Methods}, vol.~11, no.~7, p.~721, 2014.

\item A.~I. Dell, J.~A. Bender, K.~Branson, I.~D. Couzin, G.~G. de~Polavieja, L.~P.
  Noldus, A.~P{\'e}rez-Escudero, P.~Perona, A.~D. Straw, M.~Wikelski, {\em
  et~al.}, ``Automated image-based tracking and its application in ecology,''
  {\em Trends in Ecology \& Evolution}, vol.~29, no.~7, pp.~417--428, 2014.

  \item Y.~Aso, D.~Sitaraman, T.~Ichinose, K.~R. Kaun, K.~Vogt, G.~Belliart-Gu{\'e}rin, P.-Y. Pla{\c{c}}ais, A.~A. Robie, N.~Yamagata, C.~Schnaitmann, W.~J. Rowell, R.~M. Johnston, T.-T~B. Ngo, N.~Chen, W.~Korff, M.~N. Nitabach, U.~Heberlein, T.~Preat, K.~Branson, H.~Tanimoto,G.~M. Rubin,  ``Mushroom body output neurons encode valence and guide memory-based action selection in {\em {{D}}rosophila},'' {\em eLife}, vol.~3, p.~e04580, 2014.

\item F.~Amat, W.~Lemon, D.~P. Mossing, K.~McDole, Y.~Wan, K.~Branson, E.~W. Myers,
  and P.~J. Keller, ``Fast, accurate reconstruction of cell lineages from
  large-scale fluorescence microscopy data,'' {\em Nature Methods}, vol.~11,
  no.~9, p.~951, 2014.

\item D.~G. Tervo, M.~Proskurin, M.~Manakov, M.~Kabra, A.~Vollmer, K.~Branson, and
  A.~Y. Karpova, ``Behavioral variability through stochastic choice and its
  gating by anterior cingulate cortex,'' {\em Cell}, vol.~159, no.~1,
  pp.~21--32, 2014.
  
\item M.~Kabra, A.~A. Robie, M.~Rivera-Alba, S.~Branson, and K.~Branson, ``{{JAABA}}:
  An interactive machine-learning tool for automatic annotation of animal
  behavior,'' {\em Nature Methods}, vol.~10, pp.~64--67, 2013.

\item F.~Zabala, P.~Polidoro, A.~Robie, K.~Branson, P.~Perona, and M.~H. Dickinson,
  ``A simple strategy for detecting moving objects during locomotion revealed
  by animal-robot interactions,'' {\em Current Biology}, vol.~22, no.~14,
  pp.~1344--1350, 2012.

\item A.~D. Straw, K.~Branson, T.~R. Neumann, and M.~H. Dickinson, ``Multi-camera
  real-time three-dimensional tracking of multiple flying animals,'' {\em
  Journal of The Royal Society Interface}, vol.~8, no.~56, pp.~395--409, 2010.

\item K.~Branson, A.~A. Robie, J.~Bender, P.~Perona, and M.~Dickinson,
  ``High-throughput ethomics in large groups of \emph{{D}rosophila},'' {\em
  Nature Methods}, vol.~6, pp.~451--457, 2009.

\item S.~Agarwal, K.~Branson, and S.~Belongie, ``Higher order learning with graphs,''
  in {\em Proceedings of the 23rd international conference on Machine
  learning}, pp.~17--24, ACM, 2006.

\item K.~Branson and S.~Belongie, ``Tracking multiple mouse contours (without too
  many samples),'' in {\em 2005 IEEE Computer Society Conference on Computer
  Vision and Pattern Recognition (CVPR'05)}, vol.~1, pp.~1039--1046, IEEE,
  2005.

\item K.~Branson, V.~Rabaud, and S.~Belongie, ``Three brown mice: See how they run,''
  in {\em In VS-PETS Workshop at ICCV}, Citeseer, 2003.

\item G.~W. Cottrell, K.~M. Branson, and A.~J. Calder, ``Do expression and identity
  need separate representations?,'' in {\em Proceedings of the Annual Meeting
  of the Cognitive Science Society, 24 (24)}, 2002.

\end{cvenum}


\vspace{\acvSectionContentTopSkip}
\cvsection{Mentorship}
\vspace{\acvSectionContentTopSkip}
\begingroup
\begin{center}
%\fontsize{10pt}{1em}\selectfont
\fontsize{10pt}{1em}\bodyfont\upshape\selectfont
\setlength{\tabcolsep}{6pt}
\begin{tabular}{ll>{\raggedright\arraybackslash}p{1.5in}>{\raggedright\arraybackslash}p{2.5in}}
\hline
  {\em Name} & {\em Years in lab} & {\em Degree received} & {\em Current Position} \\\hline\hline
  Alice Robie & 2010-present & & Senior Scientist in my lab \\
  Mayank Kabra & 2011-2013 & & Machine learning consultant, Kabra Consulting \\
  Marta Rivera-Alba & 2013-2016& & Head of Data Science at Causal Foundry\\
  Nakul Verma & 2013-2017 & & Professor, Computer Science, Columbia U.\\
  Kelly Seagraves & 2015-2016 & & Senior Advisor in the U.S. Department of State’s Office of the Special Envoy for Critical and Emerging Technology \\
  Roian Egnor & 2015-present & & Senior Scientist in my lab\\
  Iljung (Sam) Kwak & 2015-2023 & PhD, CS, UCSD, 2019 & 3D Machine Learning Engineer at Nuwa\\
  Jiwoong (Daniel) Im & 2016-2020 & & PhD student, CS, NYU\\
  Rutuja Patil & 2017-2024 & & Software Engineer in my lab\\
  Ivan (Felip\'e) Rodriguez & 2017-2018 & M.S., CS, U.~Puerto Rico, 2019 & PhD student, Cognitive Science, Brown University \\
  Lingqi Zhang & 2023-present & & Theory Fellow with my lab \\
  Aniket Ravan & 2024-present & & Machine Learning Researcher in my lab \\
  Eyrun Eyjolfsdottir & 2024-present & & Machine learning consultant \\\hline
\end{tabular}
\end{center}
\endgroup
\vspace{.25cm}
With members of my lab, I've also co-mentored two undergraduate students and three high school students. 2/3 of my mentees have been from groups historically marginalized in STEM. As Head of Computation and Theory, I provided mentorship and advice to the other 9 Group Leaders within this Research Area. 

\vspace{\acvSectionContentTopSkip}

\cvsection{Outside the lab}

\vspace{\acvSectionContentTopSkip}
\begin{cvitems}
\item I love rock climbing, hiking, and generally being outside in the sun. My favorite is brainstorming zany machine learning and science ideas while doing these activities with other scientists. 
\item I enjoy word puzzles, particularly crossword puzzles. 
\end{cvitems}
 
\end{document}

\cvsection{Invited Talks}
\vspace{\acvSectionContentTopSkip}
\vspace{2mm}

\begin{cvitems}
\item Conference on Simulated Bodies, Keynote, September, 2023. 
\item Drosophila Neural Circuits, May, 2023.
\item University of North Carolina Neuroscience Seminar, May, 2023.
\item Bioimage Computing, ECCV, October, 2022. 
\item Machine Learning Workshop, Jackson Labs, October, 2022. 
\item Visualizing Biological Data EMBO Workshop, January, 2022. 
\item Learning Meaningful Representations of Life (LMRL) Workshop, NeurIPS, December, 2021.
\item CU Boulder BioFrontiers Seminar, October, 2021.  
\item CVPR Computer Vision for Animals Workshop, June, 2021. 
\item CVPR Multi-Agent Behavior Workshop, June, 2021.
\item World Wide Neuro/Tubingen Neuroscience Seminar Series, May, 2021.
\item HHMI Science Meeting, December, 2020.
\item Life Sciences Across the Globe, October, 2020.
\item Georgetown Biology Seminar Series, October, 2020. 
\item ECCV BioImage Computing Workshop, August, 2020. 
\item CVPR Workshop on Fine-Grained Visual Categorization, June, 2020. 
\item NEUBIAS Symposium, Keynote, February, 2020. 
\item COSYNE Workshop on Interpretable Computational Neuroscience, February, 2020. 
\item Visipedia Conference, February, 2020. 
\item Max Planck Institute Future of Neuroscience Symposium, November, 2019.
\item Janelia Junior Scientist Workshop on Theoretical Neuroscience, October, 2019. 
\item Stanley Center Symposium, September, 2019. 
\item Cajal Course on Interacting with Neural Circuits, July, 2019.
\item NSF Scholarship fund for excellence in Computer Science and Mathematics, University of Puerto Rico, Recinto de Rio Piedras, April, 2019.
\item Swartz Seminar Series, Yale, March, 2019.
\item HHMI Food for Thought, HHMI Headquarters, Februrary, 2019.
\item FENS Winter School on Innate Behavior, January, 2019.
\item NSF Machine Learning and Biology Workshop, Harvard University Physics of the Living Systems, December, 2018.
\item Center for Theoretical Neuroscience, Columbia University, November, 2018.
\item International Brain Lab Annual Meeting, May, 2018. 
\item Keynote, Simons Collaboration for the Global Brain Annual Meeting, September, 2018.
\item GRASS Fellows Seminar at Marine Biology Lab, August, 2018.
\item Cajal Course on the Behavior of Neural Systems, July, 2018.
\item International Symposium on Biomedical Imaging, April, 2018. 
\item BigNeuro Workshop, NIPS, December, 2017.
\item Janelia Coference on Emerging Tools for Acquisition and Interpretation of Whole-Brain Functional Data, November, 2017.
\item HHMI Science Meeting, October, 2017. 
\item Champalimaud Neuroscience Seminar, September, 2017. 
\item Keynote, INCF Neuroinformatics Congress, August, 2017. 
\item Kavli Workshop on Neural Circuits and Behavior of Drosophila, July, 2017.
\item Puerto Rico INBRE and COBRE Scientific Symposium, May, 2017.
\item Universisty of Pennsylvania, GRASP Seminar, April, 2017.
\item Univesity of Texas, Houston Graduate Student Invited Neuroscience Seminar, February, 2017. 
\item Workshop on High-Dimensional Neuro-Behavioral Analyses, Cosyne, February, 2017.
\item Johns Hopkins Biology Department Seminar, February, 2017.
\item Simons Collaboration on the Global Brain Quantitative Behavior Workshop, November, 2016. 
\item Society for Neuroscience Minisymposium, November, 2016. 
\item Coordinating Global Brain Projects, September, 2016.
\item HHMI Science Meeting, September, 2016. 
\item Journal of Experimental Biology Evolution of Social Behavior Workshop, March, 2016. 
\item Stanford Neuroscience Seminar, February, 2016. 
\item MIT Neurotech Symposium, November, 2015. 
\item Janelia Conference on Emerging Tools for Acquisition and Interpretation of Whole-Brain Functional Data, November, 2015. 
\item EMBO Workshop on Neural circuits and behaviour of Drosophila, June, 2015.
\item BioImage Computing Workshop, CVPR, June, 2015.
\item Biophysical Society Annual Meeting, February, 2015.
\item Neuroscience Graduate Program Seminar, Brown, October 2014. 
\item Janelia Conference on Life in the Aggregate: Mechanisms and Features of Social Dynamics, October, 2014. 
\item Measuring Behavior, August, 2014.
\item Mathematical Biosciences Institute Workshop on Analysis of Large Collections of Imaging Data, April, 2014. 
\item AAAS Neuroscience and Data Sharing Symposium, 2014. 
\item Advisory Committee to the NIH Director BRAIN Working Group, July, 2013.
\item FliACT Training Workshop, April, 2013. 
\item International Symposium on Neuroethology, August, 2012. 
\item Janelia Machine Learning Conference, May, 2012. 
\item Cold Spring Harbor Conference on High-Throughput Automated Phenotyping, April, 2012. 
\item Society for Integrative and Comparative Biology Annual Meeting, January, 2012. 
\item Annual Drosophila Research Conference, April, 2010. 
\end{cvitems}



%-------------------------------------------------------------------------------
\end{document}
